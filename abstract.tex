\begin{spacing}{2}
\begin{center}
\fontsize{24pt}{24pt}\selectfont \textbf{ABSTRACT\\}
\vspace*{1cm}
\end{center}
\fontsize{14pt}{14pt}\selectfont

The past decade has witnessed an exponential growth in the number of robots and unmanned vehicles in use by defence organisations, industries and general public for a wide variety of uses from surveillance and manufacturing to learning and entertainment. However, the effect is highly significant in the field of unmanned aerial vehicles (UAVs). Owing to the low-cost and ease of availability, the use of multi-rotor UAVs has become a common sight. With roboticists drawing inspiration for ornithopter (flapping wing) UAV designs and increasing presence of these vehicles in airspace, the idea of devising strategies for these vehicles to collaborate like a flock of birds seems both intriguing and promising.

Swarm robotics, though studied in literature for decades, is still a new field of research with implementation and establishment of reliable communication links being major challenges. The following projects provides an overview of current state of aerial robotic swarms and entails the development of a low-cost UAV swarming test-bed which can be used for testing of swarm specific algorithms. The system proposed is decentralised and scalable in nature and allows for dynamic handling of new agents in the environment.

\vspace{0.8cm}
\end{spacing}